\documentclass[pdftex, a4paper, 11pt]{article}
\usepackage[italian]{babel}
\usepackage{fancyhdr,graphicx}
\usepackage[hmargin=2cm,vmargin=2cm,a4paper]{geometry}
\usepackage{hyperref}

\pagestyle{fancy}
\lhead{\scshape Curriculum Vitae}
\rhead{\itshape Enrico Carlesso}
\rfoot{\footnotesize pag. \thepage}
\cfoot{}
\lfoot{{\footnotesize Aggiornato al:} \today}
\renewcommand{\headrulewidth}{1pt}
\renewcommand{\footrulewidth}{1pt}

\begin{document}
\vspace*{.3cm}
\begin{center}
  \rule{.8\textwidth}{1pt}\\[10pt]
  \begin{minipage}{.55\textwidth}
    \LARGE\textbf{Enrico Carlesso}\\[13pt]
    \small Via Julia, 37\\
    36060 - Romano d'Ezzelino (VI)\\[6pt]
    \textbf{e-mail: enrico@ecarlesso.org}\\
    \small \url{http://www.ecarlesso.org}\\
    \small N. Tel: +39 348 858 77 86\\[6pt]
    \small Nazionalit\'a: Italiana\\
    %% \small Data di nascita: 03.12.1984\\
    \small Stato civile: celibe\\
  \end{minipage}
  \begin{minipage}{.2\textwidth}
    \includegraphics[width=\textwidth]{foto.png}
  \end{minipage}\\[5pt]
  \rule{.8\textwidth}{1pt}
\end{center}
\vspace*{1cm}

\section*{Posizione Attuale:}


CTO in Doochoo Inc.

Iscrizione al Registro Imprese con Partita IVA n. 03472460249 con qualifica di ``Consulente nel settore delle tecnologie dell'informatica''

% Stageaire presso la divisione R\&D di M31.

\section*{Istruzione}
\begin{itemize}
\item Laurea Specialistica in Ingegneria Informatica con votazione 101/110 (Universit\`a degli studi di Padova, 2011);
\item Laurea Triennale in Ingegneria Informatica con votazione 92/110 (Universit\`a degli
  studi di Padova, 2007);
\item Diploma di Ragioneria (I.T.C.G. L. Einaudi di Bassano del Grappa,
  2003).
\end{itemize}

\section*{Esperienze}
\begin{itemize}
\item Stage presso il reparto R\&D di M31;
\item Tesi presso M31 dal titolo ``Un sistema di monitoraggio e controllo remoto per dispositivi industriali'';
\item Realizzazione di progetti web per diversi clienti;
\item Fornitura di consulenze nell'ambito di creazione e manutenzione server;
\item Partecipazione al German Open 2007, con la squadra
  dell'universit\`a di Padova nella ``Humanoids Legue'';
\item Tesi presso il Laboratorio di Sistemi Autonomi ed
  Intelligenti dell'universit\`a di Padova dal titolo ``Algoritmi
  simultanei di localizzazione e mappatura basati sulla visione artificiale'';
\item Realizzazione di diverso software opensource,
  prevalentemente in Python e C.
\end{itemize}

\section*{Compenze informatiche}
\begin{description}
\item[Web:] Ottima conoscenza dei framework RubyOnRails (Ruby), Django
  (Python) e CakePHP (php). Ottime capacit\'a in
  tutto ci\`o che riguarda il web e i diversi standard (XHTML 1.1,
  1.0 Transitional, 1.0 Strict). Ottima conoscenza
  di Javascript, specializzato nell'utilizzo delle librerie MooTools e jQuery.
  Ottima conoscenza dei maggori Database: MySQL, PostgreSQL, MongoDB, Sqlite.
\item[S.O.:] GNU/Linux, Archlinux \& Slackware {\em in primis},
  ottima conoscenza
  in ogni campo, dall'installazione, alla configurazione, alla
  creazione di software di supporto. Ottima conoscenza del linguaggio
  {\em bash} ed ottime capacit\`a di realizzazione di script
  mirato. Ottima competenza in ambito server.
\item[Programmazione:] Ottima conoscenza di Python e Ruby. Buone competenze in C e C++. 
  Esperienze di programmazione con molti altri linguaggi (Java, perl);
\item[Python:] Ampia esperienza di programmazione e ottima conoscenza
  di ambienti dedicati: pyQT, pyOpenGL, OpenCV bindings, pyGTK,
  pyserial.
\item[Elettronica:] Buone competenze di programmazione di
  microcontrollori della famiglia Atmel, esperienza di disegno di PCB,
  buona conoscenza dell'applicativo Eagle. Realizzazione di una scheda
  controllo motori per robot. Buona esperienza con la piattaforma Arduino.
\item[Altro:] \LaTeX, emacs, (py)QT, OpenGL, vim, iptables, cups,
  samba, foss, gestione di reti.
\end{description}

\section*{Esperienze Lavorative}
\begin{itemize}
\item Maggio 2011/Attualmente: CTO per Doochoo Inc.;
\item Maggio 2008/Attualmente: Libero professionista;
\item Aprile 2010/Marzo 2011: Ingegnere del software presso M31;
\item Gennaio/Maggio 2008: Assistenza hardware/software presso DV
  Service (Romano d'Ezzelino - VI);
\item Maggio/Ottobre 2007: Ricerca e sviluppo presso Zilio
  S.p.A. (Cassola - VI), con progettazione e vendita di un sistema di
  videosorveglianza basato su Linux e Videocamere ad alta definizione
  (5Mpx);
\item 1999/2004: Cameriere nei fine settimana presso Villa Razzolini
  Loredan (Asolo - TV);
\item Giugno/Ottobre 2002: Impiegato Tecnico presso ``IALC'' (Romano
  d'Ezzelino - VI).
\end{itemize}

%\section*{Progetti realizzati}
%\begin{itemize}
%\item pynokia: programma che permette l'invio di sms (e
  %altre interazioni) dal computer attraverso connessione
  %bluetooth/cavo a cellulari Nokia. {\em (Python)}\\
  %\url{http://github.com/carlesso/pynokia}
%\item videntify: programma per estrapolare informazioni da
  %file video. {\em (C)}\\
  %\url{http://www.ecarlesso.org/works/show/videntify}
%\item Freesky: scheda controllo motori da servomodellismo,
  %progettazione, assemblamento e firmware. {\em (Eagle, ARM-C)}\\
%%  \url{http://www.ecarlesso.org/index.php/freesky}
%\end{itemize}

%% \section*{Portfolio WEB}
%% \begin{description}
%% \item[cnssrl.it] Sito dell'azienda CNS srl, con catalogo prodotti
%%   e backend di amministrazione a misura del cliente. Basato su CakePHP
%%   e MooTools\\
%%   \url{http://www.cnssrl.it};
%% \item[ecarlesso.org] Sito personale, in continua crescita. Basato
%%   su CakePhP e MooTools\\
%%   \url{http://www.ecarlesso.org};
%% \end{description}

\section*{Interessi personali}
\begin{itemize}
\item Informatica generale;
\item Web e tecnologie ``cutting-edge'' come nuova esperienza per l'utente;
\item {\bf F}ree and {\bf O}pen {\bf S}ource {\bf S}oftware;
\item Matematica, con particolare attenzione all'algebra;
\item Interazione tra elettronica ed informatica;
\item Ottobre 2009 - Attualmente: GrappaLUG - Linux User Group di Bassano del Grappa;
\item Ottobre 2006 - Attualmente: Membro della Pro Loco di Romano d'Ezzelino.
\end{itemize}

\section*{Competenze Linguistiche}
\begin{description}
\item[Italiano:] Madrelingua;
\item[Inglese:] Ottimo livello di comprensione e scrittura;
\item[Tedesco:] Conoscenza didattica.
\end{description}

\section*{Nel web}
\begin{description}
\item[github.com] La maggior parte dei miei progetto risiede sulla mia pagina su github.com: \\ \mbox{\url{https://github.com/carlesso}};
\item[linkedin] La mia pagina su linkedin: \\ \url{http://www.linkedin.com/in/ecarlesso};
\item[stackoverflow.com] Il mio profilo su stackoverflow careers: \\ \url{http://careers.stackoverflow.com/ecarlesso}.
\end{description}

\vfill

%% Ai sensi della legge 675/96 (tutela della persona ed altri soggetti
%% rispetto al trattamento dei dati personali), autorizzo al trattamento
%% dei dati personali contenuti nel presente Curriculum Vitae per
%% permettere un adeguata valutazione della mia candidatura finalizzata all'assunzione.
Autorizzo il trattamento dei dati da me forniti ai sensi della legge
675/96 sulla privacy.

\vspace{1cm}

\footnotesize {Questo curriculum \`e hostato sotto {\em git}. \`E possibile scaricare la versione aggiornata:}
\begin{verbatim}
    $ git clone git://github.com/carlesso/ecarlesso_cv.git
\end{verbatim}
\end{document}
